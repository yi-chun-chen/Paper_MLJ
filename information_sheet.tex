\documentclass{article}
\usepackage[utf8]{inputenc}
\usepackage[round]{natbib}
\usepackage{graphicx}
\usepackage{color}

\newcommand{\todo}[1]{\textcolor{magenta}{#1}}
\newcommand{\yichun}[1]{\textit{\textcolor{green}{#1}}}
\newcommand{\tim}[1]{\textit{\textcolor{blue}{#1}}}
\newcommand{\mykel}[1]{\textit{\textcolor{cyan}{#1}}}

\title{Information Sheet \\ {\large Learning Discrete Bayesian Networks from Continuous Data}}
\author{\normalsize Yi-Chun Chen, Tim A. Wheeler, Mykel J. Kochenderfer}
\date{}

\begin{document}

\maketitle

\noindent
\textbf{What is the main claim of the paper?}

This paper introduces a principled Bayesian discretization method for continuous variables in Bayesian networks with quadratic complexity instead of the cubic complexity of the standard minimum-description length (MDL) technique.
In addition, this paper shows how to incorporate existing methods into the structure learning process to discretize all continuous variables while simultaneously learning Bayesian network structures.\\[0em]

\noindent
\textbf{How is this a contribution to the machine learning literature?}

Bayesian networks are an increasingly popular method for modeling uncertainty and causality.
In most cases, one assumes that the random variables in a Bayesian network are discrete, since many Bayesian network learning and inference algorithms are unable to efficiently handle continuous variables.
However, many applications require the use of continuous variables, in which case one must use Bayesian networks with continuous parametric distributions or discretize all continuous variables.

The proposed method learns discretization policies for continuous variables from continuous data while taking the Bayesian network structure into account.
Prior Bayesian methods, such as MODL \citep{Boulle_2006, Lustgarten_2011}, only consider the discretization of a single continuous variable with respect to another single discrete variable.
A common discretization method for Bayesian networks is to split continuous variables into uniform-width intervals or to use field-specific expertise.
A more principled method is the minimum description length (MDL) principle discretization \citep{Friedman_1996} which has cubic runtime in the number of data samples.

The proposed method learns the discretization with quadratic runtime in the number of data samples, allowing it to more efficiently scale to larger datasets.
Furthermore, empirical demonstrations show that the proposed method produces superior discretization policies, which was objectively evaluated using the cross-validated log-likelihood.\\[0em]

\noindent
\textbf{What is the evidence you provide to support your claim?}

The Bayesian discretization algorithm is derived from base priors and the pseudocode is given.
Experiments were conducted on three datasets from the University of California, Irvine machine learning repository~\citep{Lichman_2013}.
Each dataset possesses different qualities which affect the behavior of the discretization algorithms.

Two experiments were conducted for each dataset: one experiment in which the continuous variables are discretized with a known Bayesian network structure, and a second experiment in which the network structure is simultaneously learned along with the discretization policy.
The resulting discretization policies are given and discussed for each dataset.

Results demonstrate that the proposed method is superior to the MDL work of \cite{Friedman_1996}.
The latter tends to produce discretization policies with low numbers of intervals, and the models produced by the proposed Bayesian method have significantly higher validation log-likelihood.\\[0em]

\noindent
\textbf{What papers by other authors make the most closely related contributions, and how is your paper related to them?}

The proposed method is related to \cite{Boulle_2006} and \cite{Lustgarten_2011}.
These two works discuss the discretization of a continuous variable given a single class variable using a principled Bayesian method.
This paper is an extension of these two works to discretization in a general Bayesian network, where the discretization of a particular variable requires considering that variable's Markov blanket.\\[0em]

\noindent
\textbf{Have you published parts of your paper before?}

No parts of this paper have been previously published.

\bibliographystyle{spbasic}
\bibliography{my_bib}
\end{document}
